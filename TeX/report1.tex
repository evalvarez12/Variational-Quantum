% !TeX encoding = UTF-8
% !TeX spellcheck = en_US
% !TeX root = report1.tex


\documentclass[conference]{IEEEtran}
\usepackage{blindtext, graphicx}
\usepackage{amsmath, amsthm, amssymb, amsfonts, nicefrac}



\begin{document}

\title{Monte Carlo Integration and Variational Quantum Mechanics}


\author{\IEEEauthorblockN{Eduardo Villase\~nor}
\IEEEauthorblockA{TU Delft (4624548)}
\and
\IEEEauthorblockN{Ren\'e Vollmer}
\IEEEauthorblockA{TU Delft (4630807)}
\and
\IEEEauthorblockN{Robbie Elbertse}
\IEEEauthorblockA{TU Delft (4078039)}}


\maketitle


\begin{abstract}

Monte Carlo simulations in combination with the variational method is a powerful tool for numerically finding ground states of quantum systems that don't have an analytic solution. In this work we examine and implement this method to approximate the ground states and the associated energies for a one-dimensional harmonic oscillator, a Hydrogen atom and a Helium atom. In order to achieve that goal, different numerical integration methods are examined. In the case of the harmonic oscillator and the hydrogen atom, relative errors in energy smaller than $10^{-5}$ could easily be obtained. For helium much larger errors were observed, since the inter-electron interaction was only approximated with a very simple test-function. Still, the results show that our implementation of the Monte Carlo methods can produce results that are comparable to other methods, even for multi-electron systems.


\end{abstract}

\IEEEpeerreviewmaketitle



\section{Introduction}

Monte Carlo simulation have long been a value tool to solve integration problems, among others. While essentially done through means of computers, there is an $18^{th}$ century precursor to this which nicely illustrates the idea of Monte Carlo simulation. In 1733, Georges-Louis Leclerc de Buffon had a wooden floor made up parallel beams; thus creating a plane with parallel lines. If he were to drop down needles on the floor, he wondered what the probability would be of any needle crossing one of the parallel lines \cite{Buffon}. Since this is dependent on the angle the needle makes with the lines, it comes to no surprise there this is a factor of pi in the probability. While his intentions were not to find the value of pi, this experiment was later done to indeed find the value of pi, in 1901 by Mario Lazzarini where after 3408 tosses he found the value of pi up to 7 digits. \cite{Lazzarini}. \\

As of the 1930s the method started to gain more popularity, especially with the advent of the computer. It comes down to using a stochastic method to determine a non-stochastic value. For example, integrals with no known analytical solution can be approximated best by using a random sample of points, rather than a periodic or otherwise predetermined set of points. A cosine wave would be poorly integrated by taking points only where the phase is zero. In this case we are aware of the poor approximation, but with more difficult integrals we cannot always know this in advance, and so a random sample will provide more accurate results. In this paper we will attempt to find the ground state (!!) energy (!! just this or also the wave function?) of a helium atom. Since this involves integrals over the product of wave functions with a Hamiltonian, we will use Monte Carlo simulation to approximate our solution. 

By finding the ground state energy using integrals, we have resorted to the variational method for quantum mechanics. In this method we look at the Hilbert space of a Hamiltonian which contains all the possible states a wave function can have in accordance to that Hamiltonian. We then restrict ourselves to a subset of this space, and try to optimize our solution; for example by finding the ground state energy. The restriction we take is that we use so-called trial wave functions, which are wave functions of a predetermined format with some unknown parameters, and we try to optimize the solution in this parameter space. It turns out that \textit{the} ground state wave function will always yield the lowest ground state energy. So when comparing two trial wave functions, the one with the lowest energy will be closer to the \textit{actual} ground state wave function. In this way we can approximate the ground state (!!) wave function (!! or just the energy?). \cite{AdvStatMech}

We will first introduce the reader to all the necessary equations in chapter 2, followed by a description of the implementation and some preliminary results in chapter 3. We then describe our results and discuss this in the light of expectations, in chapter 4. Finally, in chapter 5 we will draw conclusions and suggest options for further research.


\section{Making a Monte Carlo integrator}


First, we explored different methods
in order to gain a better insight on how to create a better general integrator. We tested 4 methods, where each differs in how
the points in the domain are chosen to evaluate the integral.
The integration schemes \cite{MCmethods} used are, see figure \ref{BoxPlotter}:
\begin{enumerate}
  \item Uniform Static - The entire domain is divided by a uniform mesh were each point is taken in a coordinate is the grid.
  \item Uniform Adaptive - The domain is divided uniformly in square boxes, and for each box a mesh grid is created
  specified by a density value which determines the total number of points inside that box. The density parameter
  is obtained by performing successive iterations, on which the density of each sub-box is set by the total value of the integral inside
  the box.
  \item Stratified Static - The entire domain is divided in a uniform mesh, where for each quadrant of the mesh
  a single point is chosen randomly inside.
  \item Stratified Adaptive - Basically a combination of both methods adaptive and stratified.
\end{enumerate}


\begin{figure*}[ht]
  \begin{center}
  \includegraphics[scale=1 ]{graphs/BoxPlotter.pdf}\label{BoxPlotter}
  \caption{Illustrative examples of (from left to right) a uniform static, uniform adaptaive, stratified static and stratified adaptive integration schemes. The lines create boxes within which an internal grid is applied in the case of static integration schemes. Notice that in all cases the number of points is the same, which in the case of adaptive static causes us to end with some points which do not fit in a grid. To accommodate for this we distributed them randomly within their respective box. }
  \label{fig:int}
  \end{center}
\end{figure*}

To compare each method we used a test function on which we evaluated the integrals.
This test function consisted of a ring with constant value in a 2-dimensional domain, from which
we can know the exact value of the integral easily. The results
of the error obtained for the different methods as a function of the number of points used is shown in fig. \ref{mc_errs},
 where 500 repetitions where used for each number of points and in the case of adaptive methods the entire integration
 domain was divided into 25 boxes. As
expected the best method is the adaptive stratified which has also gives the smallest standard deviation from the error as seen in the error bars.
Additionally it is important to distinguish the steps in the error of the uniform static
method which are the systematic error introduced by the regular mesh. This shows the entire point of doing Monte Carlo integration!
\begin{figure*}[ht]
  \begin{center}
  \includegraphics[scale=1 ]{graphs/integration_test_ring.pdf}
  \caption{Performance comparison of different numeric integration methods. The absolute value of the difference between the numerically calculated result and the analytical answer plotted against the number of integration points used. The used function is a two-dimensional ring (see text).}
  \label{fig:int}
  \end{center}
\end{figure*}


\section{Variational}
 Now we focus on using the 


% !TeX encoding = UTF-8
% !TeX spellcheck = en_US
% !TeX root = report1.tex


\section{Conclusion}
Here we conclude how awesome we all are.












\bibliographystyle{plain}
\bibliography{report}


\end{document}
