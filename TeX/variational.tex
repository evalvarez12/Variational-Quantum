% !TeX encoding = UTF-8
% !TeX spellcheck = en_US
% !TeX root = report1.tex

\section{Results}
%Now we focus on using the integration to solve the problem established in the introduction.
Now we will present the results obtained of using the Monte Carlo integration and the variational
method to find the ground state energy on each of the problems already established. In every case
we will compare
our results with the reference values found in \cite{JosBook}.

\subsection{Harmonic Oscillator}
Two simulations were performed with 50 iterations, a damping factor of $\gamma = 3\cdot 10^{-5}$,
an initial values of $\alpha_i = 3$ and $0.01$  and 1,000 test points over a domain with length 4,
subdivided into 20 sub-domains for the adaptive grid. In figure \ref{fig:Ho_it}
we show the values obtained for the energy and for $\alpha$ as a function of every iteration performed,
om which the parameter $\alpha$ is changing according to \ref{eq:minimizer},
for two different values of $\alpha_i$. We can see how  convergence values in both energy and $\alpha$ are
reached with a small number of iterations, which shows that the methods used are efficient. In figure
 \ref{fig:Ho_rel} we can see how the energy behaves as a function on $\alpha$ where there is a clear
minimum value at $0.5$ where each of the executions of our program meet each other.

The averages over the last ten iterations yielded an
error in the energy of $err(E)=E_{MC}-E_{REF}\simeq 1.431\cdot 10^{-9}$ and a error in the parameter of $err(\alpha)=\alpha - \alpha_{REF} \simeq 6.375\cdot 10^{-6}$ for the starting value of $\alpha_i=3$. For $\alpha_i = 0.01$ we obtained $err(\alpha)=\alpha - \alpha_{REF} \simeq -4.468 \cdot 10^{-3}$ and $err(E)=E_{MC}-E_{REF}\simeq 2.331\cdot 10^{-5}$. 
\begin{figure}[th]
	\begin{center}
		\includegraphics[scale=0.9]{graphs/ho-e-alpha-iterations.pdf}
		\caption{
			Results of the simulation of a one-dimensional harmonic oscillator. Energy and $\alpha$ for two different $\alpha_i$ are plotted against the number of iterations. For the parameters used for the simulation, please refer to the text.
			%Calculated $E$ as a function of $\alpha$ using two different starting values for$\alpha_0$. Notice that $\alpha$ converges monotonically (increasing or decreasing) while this is not necessarily true for the energy due to the integration not being exact. Nonetheless, a very accurate result can be found: $E_{\alpha_0 = 0.5} =  $ and $E_{\alpha_0 = 0.5} =  $.
			}
		\label{fig:Ho_it}
	\end{center}
\end{figure}
\begin{figure}[th]
	\begin{center}
		\includegraphics[scale=0.9]{graphs/ho-e-alpha.pdf}
		\caption{
			Results of the simulation of a one-dimensional harmonic oscillator. The obtained energy is plotted as a function of $\alpha$ for two initial values of $\alpha_i$. For the parameters used for the simulation, please refer to the text.
			%Calculated $E$ as a function of $\alpha$ using two different starting values for$\alpha_0$. Notice that $\alpha$ converges monotonically (increasing or decreasing) while this is not necessarily true for the energy due to the integration not being exact. Nonetheless, a very accurate result can be found: $E_{\alpha_0 = 0.5} =  $ and $E_{\alpha_0 = 0.5} =  $.
		}
		\label{fig:Ho_rel}
	\end{center}
\end{figure}


\subsection{Hydrogen Atom}
Using the parameters
$\gamma = 5\cdot 10^{-4}$, $\alpha_i = 3 $ with 5,000
test points over $7$ boxes, again about 40 iterations were needed to reach convergence
values of $\alpha \approx 1$ and $E_{MC} \approx - 1/2$, where the error obtained after 160 iterations is
$err(\alpha) = \alpha-\alpha_{0} \simeq 5 \cdot 10^{-10}$. %That is, within 160 iterations we can estimate the value $\alpha=1$ up to ten digits and find the appropriate energy for it.


\subsection{Helium Atom}
Similarly to the previous case we calculated $E$ as a function of $\alpha$
using two different starting values for $\alpha_i$, see figure~\ref{fig:He_it}.
After 40 iterations, a stable result is obtained. Averaging over the last ten results yields
$E_{\alpha_i = 0.01} = -2.82225\pm 0.0009$ at $\alpha = 0.04822 \pm 0.00021 $ and $E_{\alpha_i = 0.5} =  -2.82212\pm 0.00156$ at $\alpha=0.04902\pm 0.00043$, which
compare well with the optimum value achieved by this method of $-2.8781 \pm 0.0005$, the Hartree-Fock value of $-2.8617$,
the DFT value of $-2.83$ and the exact value of $-2.9307$ (see \cite{JosBook}).

\begin{figure}[th]
  \begin{center}
  \includegraphics[scale=1 ]{graphs/he-e-alpha-iterations.pdf}
  \caption{
	Results of the simulation of the Helium atom. Energy and $\alpha$ for two different $\alpha_i$ are plotted against the number of iterations. For the parameters used for the simulation, please refer to the text.
  	% Calculated $E$ as a function of $\alpha$ using two different starting values for$\alpha_0$.
		%  Notice that $\alpha$ converges monotonically (increasing or decreasing) while this is not necessarily
		%   true for the energy due to the integration not being exact. Nonetheless, a very accurate result
		% 	 can be found: $E_{\alpha_0 = 0.5} =  $ and $E_{\alpha_0 = 0.5} =  $
  	%TODO: This was already mentioned in the text. I should only be in one place!
  	}
  \label{fig:He_it}
  \end{center}
\end{figure}
