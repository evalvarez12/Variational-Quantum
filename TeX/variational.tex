% !TeX encoding = UTF-8
% !TeX spellcheck = en_US
% !TeX root = report1.tex

\section{Variational}
%Now we focus on using the integration to solve the problem established in the introduction.
Now we will present the results obtained of using the Monte Carlo integration and the variational
method to find the ground state energy on each of the problems already established.

\subsection{Harmonic Oscillator}
Two simulations were performed with 50 iterations, a damping factor of $\gamma = 10^{-5}$, an initial value of $\alpha_i = 3~/~ 0.01$  and 1,000 test points over a domain with length 4, subdivided into 20 sub-domains for the adaptive grid. In the final iteration we obtained an  error in the energy of $E_{MC}-E_0 \simeq 1.63\cdot 10^{-9}$,
and a error in the parameter of $\alpha - \alpha_0 \simeq 7.6455\cdot 10^{-6}$. In figure \ref{fig:Ho_it}
we show the values obtained for the energy and for $\alpha$ as a function of every iteration performed, for two different values of $\alpha_i$. We can see how  minimum values in both energy and $\alpha$ are reached with a small number of iterations, which shows that the methods used are efficient. In figure \ref{fig:Ho_rel} we can see how the energy behaves as a function on $\alpha$ where there is a clear minnum value at $0.5$ as expected.
\begin{figure}
	\begin{center}
		\includegraphics[scale=0.9]{graphs/ho-e-alpha-iterations.pdf}
		\caption{
			Energy and alpha as a function of iteration number. %TODO
			%Calculated $E$ as a function of $\alpha$ using two different starting values for$\alpha_0$. Notice that $\alpha$ converges monotonically (increasing or decreasing) while this is not necessarily true for the energy due to the integration not being exact. Nonetheless, a very accurate result can be found: $E_{\alpha_0 = 0.5} =  $ and $E_{\alpha_0 = 0.5} =  $.
			}
		\label{fig:Ho_it}
	\end{center}
\end{figure}
\begin{figure}
	\begin{center}
		\includegraphics[scale=0.9]{graphs/ho-e-alpha.pdf}
		\caption{
			Energy plotted as a function of alpha. %TODO
			%Calculated $E$ as a function of $\alpha$ using two different starting values for$\alpha_0$. Notice that $\alpha$ converges monotonically (increasing or decreasing) while this is not necessarily true for the energy due to the integration not being exact. Nonetheless, a very accurate result can be found: $E_{\alpha_0 = 0.5} =  $ and $E_{\alpha_0 = 0.5} =  $.
		}
		\label{fig:Ho_rel}
	\end{center}
\end{figure}


\subsection{Hydrogen Atom}
 The convergence on this seems to be quite fast. Using the parameters $\gamma = 5\cdot 10^{-4}$, $\alpha_i = 3 \text{~/~}0.01$ with 5,000 test points over $7$ boxes, we need about 160 iterations to reach $\alpha-\alpha_0 \simeq 5 \cdot 10^{-10}$. That is, within 160 iterations we can estimate $\alpha$ up to ten digits and find the appropriate energy for it.


\subsection{Helium Atom}

Similarly to the previous case, we solved it changing $\alpha$. Using two different starting values $\alpha_i$ we convergence to the same values for $\alpha$ and $E_{MC}$, see figure~\ref{fig:He_it}. The values we find for the energy after just 50 iterations are $E_{\alpha_i = 0.01} =  $ and $E_{\alpha_i = 0.5} =  $, which compare well with the optimum value achieved by this method of $-2.8781 \pm 0.0005$, the Hartree-Fock value of $-2.8617$, the DFT value of $-2.83$ and the exact value of $-2.9307$ (see \cite{JosBook}). %TODO

\begin{figure}
  \begin{center}
  \includegraphics[scale=1 ]{graphs/he-e-alpha-iterations.pdf}
  \caption{
  	Calculated $E$ as a function of $\alpha$ using two different starting values for$\alpha_0$. Notice that $\alpha$ converges monotonically (increasing or decreasing) while this is not necessarily true for the energy due to the integration not being exact. Nonetheless, a very accurate result can be found: $E_{\alpha_0 = 0.5} =  $ and $E_{\alpha_0 = 0.5} =  $
  	%TODO: This was already mentioned in the text. I should only be in one place!
  	}
  \label{fig:He_it}
  \end{center}
\end{figure}
