% !TeX encoding = UTF-8
% !TeX spellcheck = en_US
% !TeX root = report1.tex

\section{Variational}
 Now we focus on using the integration to solve the problem established in the introduction.
 Using the Monte Carlo integration we focused on three different problems, the harmonic oscillator,
 hydrogen atom and helium atom.

\subsection{Harmonic Oscillator}
  For the harmonic oscillator the trial wave function we choose is
  \begin{align}
    \Psi_T = e^{-\alpha x^2}
  \end{align}

  The Hamiltonian is
  \begin{align}
    \hat{H} = \frac{-1}{2}\frac{\partial^2}{ \partial x^2} + \frac{1}{2} x^2
  \end{align}
  
  Then the local energy would then be  
  \begin{align}
    E_0 = \alpha + x^2(\frac{1}{2} - 2\alpha^2)
  \end{align}  
 	
  Notice that based on the local energy we can already determine that $\alpha = \frac{1}{2}$ gives us the ground state ($E_L = E_0 = \frac{1}{2}$), because in that case the variance of the local energy is equal to zero. It is nonetheless a good test for our algorithm to see if we can find the expected value of $E_0 = \frac{1}{2}$ using Monte Carlo integration. 
  
  [insert here results]
  
\subsection{Hydrogen Atom}
We will now consider the Hydrogen atom, where will we use the trial function 

  \begin{align}
    \Psi_T = e^{-\alpha r}
  \end{align}

on the following Hamiltonian:
  
  \begin{align}
    \hat{H} = \frac{-1}{2}\nabla^2 - \frac{1}{r}
  \end{align}
  
This leads the a local energy of

  \begin{align}
    E_L(r) = - \frac{1}{r} - \frac{1}{2}\alpha(\alpha - \frac{2}{r})
  \end{align}
  
Here, too, we notice that from inspection it is clear that $\alpha = 1$ yields the ground state energy of $E_L = E_0 = -\frac{1}{2}$. 

\subsection{Hydrogen Atom}
Finally we consider the Hydrogen atom, where will we use the trial function 

  \begin{align}
    \Psi_T (\textbf{r}_1,\textbf{r}_2) = e^{2r_1}e^{2r_2}e^{\frac{r_{12}}{2(1+\alpha r_{12}}} 
  \end{align}

where $r_{12} = |\textbf{r}_1 - \textbf{r}_2 |$. This trial function is applied to the following Hamiltonian:
  
  \begin{align}
    \hat{H} = -\frac{1}{2}(\nabla_{r_1}^2 + \nabla_{r_3}^2 + 2\nabla_{r_1}\cdot \nabla_{r_2}) - \frac{1}{r}
  \end{align}
  
This leads a 'beautiful' equation for the local energy:

  \begin{align}
    E_L(\textbf{r}_1,\textbf{r}_2) = -4  + (\hat{\textbf{r}}_1 - \hat{\textbf{r}}_2) \cdot (\textbf{r}_1 - \textbf{r}_2) \frac{1}{r_{12}(1+\alpha r_{12})^2} -  \frac{1}{r_{12}(1+\alpha r_{12})^3} - \frac{1}{r_{12}(4(1+\alpha r_{12}))^4} + \frac{1}{r_{1,2}}   \end{align}
  
