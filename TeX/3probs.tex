\section{Variational Problems}
In this section we specify the problems in quantum mechanics we attempt to solve,
all of these problems have well known solutions and thus we can gauge how accurate our variational method is.

\subsection{Harmonic Oscillator}
The Hamiltonian of the one-dimensional harmonic oscillator is
\begin{align*}
  \hat{H} = -\frac{1}{2}\frac{\partial^2}{ \partial x^2} + \frac{1}{2} x^2 \text{~.}
\end{align*}
We choose the trial wave function as
  \begin{align*}
    \Psi_T = e^{-\alpha x^2} \text{~,}
  \end{align*}
from which we can, together with eq.~\eqref{eq:local_energy}, derive the local energy
  \begin{align*}
    E_L = \alpha + x^2 \left( \frac{1}{2} - 2\alpha^2 \right) \text{~.}
  \end{align*}

Notice that based on the local energy we can already determine
that $\alpha = \frac{1}{2}$ gives us the ground state ($E_L = E_0 = \frac{1}{2}$),
because in that case the variance of the local energy is equal to zero.
It is nonetheless a good test for our algorithm to see if we can find the expected
value of $E_0 = \frac{1}{2}$.



\subsection{Hydrogen Atom}
We will now consider the Hydrogen atom, where the Hamiltonian is
\begin{align*}
  \hat{H} = -\frac{1}{2}\nabla^2 - \frac{1}{r}.
\end{align*}
The choosen trial wave function is,
  \begin{align*}
    \Psi_T = e^{-\alpha r}.
  \end{align*}
This leads the a local energy
  \begin{align*}
    E_L(r) = - \frac{1}{r} - \frac{1}{2}\alpha  \left( \alpha - \frac{2}{r} \right)
  \end{align*}

Here, too, we notice that from inspection it is clear that $\alpha = 1$ yields the ground state energy of $E_L = E_0 = -\nicefrac{1}{2}$.


\subsection{Helium Atom}
Finally we consider the Hamiltonian for the Helium atom,
\begin{align*}
  \hat{H} = -\frac{1}{2}(\nabla_{r_1}^2 + \nabla_{r_2}^2 + 2\nabla_{r_1}\cdot \nabla_{r_2}) - \frac{1}{r}.
\end{align*}
With the trial function
  \begin{align*}
    \Psi_T (\textbf{r}_1,\textbf{r}_2) = e^{2r_1}e^{2r_2}e^{\frac{r_{12}}{2(1+\alpha r_{12}}}.
  \end{align*}
where $r_{12} =\left| \textbf{r}_1 - \textbf{r}_2 \right|$.
This gives the expression for the local energy:
  \begin{align*}
    E_L(\textbf{r}_1,\textbf{r}_2) &= -4  +
     \frac{(\hat{\textbf{r}}_1 - \hat{\textbf{r}}_2) \cdot (\textbf{r}_1 - \textbf{r}_2)}{r_{12}(1+\alpha r_{12})^2} \\
     &~~ -   \frac{1}{r_{12}(1+\alpha r_{12})^3}  - \frac{1}{r_{12}(4(1+\alpha r_{12}))^4} + \frac{1}{r_{1,2}}
  \end{align*}
