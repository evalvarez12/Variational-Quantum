\section{Introduction}

Monte Carlo simulation have long been a value tool to solve integration problems, among others. While essentially done through means of computers, there is an $18^{th}$ century precursor to this which nicely illustrates the idea of Monte Carlo simulation. In 1733, Georges-Louis Leclerc de Buffon had a wooden floor made up parallel beams; thus creating a plane with parallel lines. If he were to drop down needles on the floor, he wondered what the probability would be of any needle crossing one of the parallel lines \cite{Buffon}. Since this is dependent on the angle the needle makes with the lines, it comes to no surprise there this is a factor of pi in the probability. While his intentions were not to find the value of pi, this experiment was later done to indeed find the value of pi, in 1901 by Mario Lazzarini where after 3408 tosses he found the value of pi up to 7 digits. \cite{Lazzarini}. \\

As of the 1930s the method started to gain more popularity, especially with the advent of the computer. It comes down to using a stochastic method to determine a non-stochastic value. For example, integrals with no known analytical solution can be approximated best by using a random sample of points, rather than a periodic or otherwise predetermined set of points. A cosine wave would be poorly integrated by taking points only where the phase is zero. In this case we are aware of the poor approximation, but with more difficult integrals we cannot always know this in advance, and so a random sample will provide more accurate results. In this paper we will attempt to find the ground state (!!) energy (!! just this or also the wave function?) of a helium atom. Since this involves integrals over the product of wave functions with a Hamiltonian, we will use Monte Carlo simulation to approximate our solution. 

By finding the ground state energy using integrals, we have resorted to the variational method for quantum mechanics. In this method we look at the Hilbert space of a Hamiltonian which contains all the possible states a wave function can have in accordance to that Hamiltonian. We then restrict ourselves to a subset of this space, and try to optimize our solution; for example by finding the ground state energy. The restriction we take is that we use so-called trial wave functions, which are wave functions of a predetermined format with some unknown parameters, and we try to optimize the solution in this parameter space. It turns out that \textit{the} ground state wave function will always yield the lowest ground state energy. So when comparing two trial wave functions, the one with the lowest energy will be closer to the \textit{actual} ground state wave function. In this way we can approximate the ground state (!!) wave function (!! or just the energy?). \cite{AdvStatMech}

We will first introduce the reader to all the necessary equations in chapter 2, followed by a description of the implementation and some preliminary results in chapter 3. We then describe our results and discuss this in the light of expectations, in chapter 4. Finally, in chapter 5 we will draw conclusions and suggest options for further research.